\documentclass[12pt,a4paper,spanish]{amsbook}
\usepackage[T1]{fontenc}
\usepackage[latin1]{inputenc}
\usepackage{babel}
\usepackage{graphics}
\usepackage{algorithm}
\usepackage{algorithmic}

\begin{document}

\begin{algorithm}
\caption{Insertar: Inserta los sufijos supermaximales de $S$ al �rbol.}
\label{Algoritmo: Insertar}
\begin{algorithmic}
\FORALL {Sufijos $\epsilon$ de $S$}
\STATE Llegar al nodo $N$ correspondiente a un prefijo $\pi$ de $\epsilon$, tal que $\pi$ es $\epsilon$ o $N$ es una hoja del �rbol.
\STATE Agregar $\pi$ a $N$.
\IF {$\pi$ no es $\epsilon$}
\STATE Expandir el nodo $N$ seg�n el s�mbolo $s$, donde $\pi s$ es prefijo de $\epsilon$.
\ENDIF
\ENDFOR
\end{algorithmic}
\end{algorithm}

\begin{algorithm}
\caption{Expandir: Expande un nodo $N$ seg�n el s�mbolo $s$.}
\label{Algoritmo: Expandir}
\begin{algorithmic}
\IF{ Cantidad de subsecuencias del nodo $N$ que continuan con el s�mbolo $s$ es mayor o igual a $c$ }
\STATE Mover todas las subsecuencias del nodo $N$ que se continuan con el s�mbolo $s$ al nodo correspondiente al eje $s$.
\FORALL{ Rama $i$ del nodo $M$ de la rama $s$ }
\STATE Expandir el nodo $M$ seg�n el s�mbolo $i$.
\ENDFOR
\ENDIF
\end{algorithmic}
\end{algorithm}

\end{document}