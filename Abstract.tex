\vspace*{3cm}

\begin{center}
\bf{Resumen}
\end{center}

El objetivo de esta tesis es proponer una nueva metodolog�a para identificar secuencias repetidas exactas en un conjunto de fragmentos de una secuencia de ADN, y as� permitir el acceso a estos elementos durante su proceso de secuenciamiento aleatorio. 
Surgido de la limitaci�n de conocer el genoma completo del \emph{Trypanosoma cruzi}, para luego as� obtener todas sus repeticiones para su eventual anotaci�n, este trabajo introduce la teor�a de \emph{�lgebra de repeticiones}. Permitiendo as� relacionar correctamente una nueva estructura de datos, $AS^2$, con el problema de encontrar las repeticiones supermaximales de un conjunto de subsecuencias de una secuencia.

\vfill

\begin{center}
\bf{Abstract}
\end{center}

These work introduces a new method to identify exact DNA repeats from a set of fragments of a DNA sequence, to allow queries about them while the random sequence strategy is in progress. 
The limits to know the complete genome of \emph{Trypanosoma cruzi}, do not permit to researchers to annotate all repeats in the known way. In order to solve this problem, these thesis introduces the \emph{repeats algebra} theory to relate a new data structure, $AS^2$, to the \emph{supermaximal repeats in a set of subsequences of a sequence search problem}.

\vspace*{3cm}
